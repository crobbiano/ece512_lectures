%----------- Necessary Style Preamble -----------%

\documentclass[mathserif, 10pt]{beamer} %

\usepackage[framesassubsections]{beamerprosper}
\usepackage{beamerthemesplit} % Activate for custom appearance
\usepackage[english]{babel}
\usepackage[latin1]{inputenc}
\usepackage{amsmath, hyperref, subfigure, multirow, rotating}
\usepackage{epstopdf}
\usepackage{verbatim}
\usepackage{listings}
\usepackage{color}
\usepackage{esint}
\usepackage{mathrsfs}

%\usepackage{enumitem}

\usetheme{Frankfurt}
\usecolortheme{default}
\usecolortheme[rgb={0.1,0.4,0.0}]{structure} % CSU color style
%\usecolortheme[rgb={0.541,0.149,0.196}]{structure} % BME color style

\setbeamersize{text margin left=0.5cm}
\setbeamersize{text margin right=0.5cm}

\setbeamercovered{transparent}

\usepackage{remreset}
\makeatletter
\@removefromreset{subsection}{section}
\makeatother
\setcounter{subsection}{1}

\makeatletter
\newcommand\xleftrightarrow[2][]{\ext@arrow 0099{\longleftrightarrowfill@}{#1}{#2}}
\def\longleftrightarrowfill@{\arrowfill@\leftarrow\relbar\rightarrow}
\makeatother


\newcommand\Wider[2][3em]{%
\makebox[\linewidth][c]{%
  \begin{minipage}{\dimexpr\textwidth+#1\relax}
  \raggedright#2
  \end{minipage}%
  }%
}

%----------- Math Definitions -----------%

\definecolor{Blue}{rgb}{0,0,1}
\def\ci{\perp\!\!\!\perp}
\def\a{\mathbf{a}}
\def\d{\mathbf{d}}
\def\e{\mathbf{e}}
\def\f{\mathbf{f}}
\def\g{\mathbf{g}}
\def\b{\mathbf{b}}
\def\q{\mathbf{q}}
\def\v{\mathbf{v}}
\def\x{\mathbf{x}}
\def\y{\mathbf{y}}
\def\u{\mathbf{u}}
%\def\z{\mathbf{\mathfrak{z}}}
\def\z{\mathbf{z}}
\def\D{\mathbf{D}}
\def\S{\mathbf{S}}
\def\X{\mathbf{X}}
\def\Z{\mathbf{Z}}
\def\1{\raisebox{.5pt}{\textcircled{\raisebox{-.9pt} {1}}}}
\def\2{\raisebox{.5pt}{\textcircled{\raisebox{-.9pt} {2}}}}
\def\3{\raisebox{.5pt}{\textcircled{\raisebox{-.9pt} {3}}}}
\def\balph{\boldsymbol\alpha}
\def\bmu{\boldsymbol\mu}
\def\btheta{\boldsymbol\theta}
\def\bomega{\boldsymbol\omega}
\def\bDelta{\boldsymbol\Delta}

%----------- Title Page Parameters -----------%
\title[Digital Control \& Digital Filters]{Digital Controls \& Digital Filters \\ Lectures 9 \& 10}
\author[M.R. Azimi]{M.R. Azimi, Professor}
\institute[CSU-ECE]{Department of Electrical and Computer Engineering \\ Colorado State University}
\date{Spring 2017}

\logo{\includegraphics[height=0.5cm]{csu-logo.png}}

\newcommand{\unt}[1]{ \mathrm{\ #1}}

\begin{document}

%----------- slide --------------------------------------------------%

\frame{\titlepage}



\section{Ideal Sampler}

%----------- slide --------------------------------------------------%
\frame
{

\small
\renewcommand{\theenumi}{\alph{enumi}}
\frametitle{Relationship between $E(s)$, $E^*(s)$ and $E(j\omega)$}

%\textcolor{red}{Relationship between $E(s)$, $E^*(s)$ and $E(jw)$}\\
Recall:  $e^*(t) =e(t) \delta_T(t)$ \vspace{.05in}

Using modulation property or convolution in $s$-domain gives:: \\ %\vspace{.1in}

$E^*(s) = \frac{1}{2\pi j} [E(s)* \Delta_T(s)] $\\ \vspace{.1in}
or
\vspace{.1in}
$E^*(s) = \frac{1}{2\pi j} \underbrace{\int_{c-j\omega}^{c+j\omega} E(\xi) \Delta_T(s-\xi)d\xi}_{\makebox{convolution integral}}$ \\
where $c$:  real number larger than the real parts of poles of $E(s)$ \\ \vspace{.1in}
\vspace{.1in}
But
 $\Delta_T(s) = \mathscr{L}\{\delta_T(t)\}=\mathscr{L}\{\delta(t) + \delta(t-T)+\delta(t-2T)+\hdots \}$ \\ \vspace{.05in}
$= 1+e^{-Ts}+e^{-2Ts} + \hdots =\frac{1}{1-e^{-Ts}}$

It can be shown using the residue theorem:\\
$E^*(s) = \sum\limits_{\makebox{poles of $E(\xi)$}} \makebox{Residues of } E(\xi) \frac{1}{1-e^{-T(s-\xi)}}$ \hspace{.1in} \1

\vspace{.1in} Or for $e^{Ts} = z$ \\
$E(z) = \sum\limits_{\makebox{poles of $E(\xi)$}} \makebox{Residues of } E(\xi) \frac{1}{1-z^{-1}e^{T\xi}}$ \hspace{.1in} \1'





}

%----------- slide --------------------------------------------------%
\frame
{

\normalsize

\frametitle{Relationship between $E(s)$, $E^*(s)$ and $E(jw)$}

\vspace{-1in}
\textcolor{red}{Special case:}\\
If $E(s)$ has real and distinct poles $p_i$'s.  i.e.\\
$E(s) = \frac{N(s)}{D(s)}, ~~~D(s) = (s-p_1)(s-p_2) \hdots (s-p_N)$\\
then \1 becomes:\\
\2  $E^*(s) = \sum\limits_{n=1}^N \frac{N(p_n)}{D'(p_n)}\frac{1}{1-e^{-T(s-p_n)}}$\\
where \\
 $D'(s) = \frac{dD(s)}{ds}\implies E(z) = \sum\limits_{n=1}^N \frac{N(p_n)}{D'(p_n)} \frac{1}{1-z^{-1}e^{Tp_n}}$ \2' \\% \vspace{.1in}

Recall: $\Delta_T(s) = \frac{1}{1-e^{-Ts}}$ with poles $e^{-Ts}=1$ \\
Thus, $Ts = 2\pi jn, \implies s_n = j n \omega_s $ where $\omega_s=\frac{2\pi}{T}$ \\
This gives \3  $E^*(s) = \frac{1}{T} \sum\limits_{n=-\infty}^{\infty} E(s+jn\omega_s)$\\
 For $s=j\omega$:\\
\3'$E^*(\omega) = \frac{1}{T}\sum\limits_{n=-\infty}^{\infty} E(\omega +n \omega_s)$\\% \vspace{.1in}
\textbf{Note: }Equation \3 is valid only when $e(0) = 0$
otherwise add $\frac{e(0)}{2}$.

\vspace{-2.1in}
\hspace{3.1in}
\includegraphics[width=.3\linewidth]{./Figures/im_poles.png} \vspace{-1in}


}

%----------- slide --------------------------------------------------%
\frame
{

\normalsize

\frametitle{Examples}

\textcolor{red}{ Example 1:}\\

Let $e(t) = u_s(t)$.  Find $E(z)$ for an ideal sampler with period $T$.\\ \vspace{.05in}

Method 1: (3 steps)\\
$e^*(t) = \sum\limits_{n=0}^{\infty} e(nT) \delta(t-nT) = \sum\limits_{n=0}^\infty \delta(t-nT)$\\
$E^*(s) = \mathcal{L}[e^*(t)] = \sum\limits_{n=0}^{\infty} e^{-nTs} = \frac{1}{1-e^{-Ts}}$\\
$E(z) = E^*(s)|_{s=\frac{1}{T}ln(z)} = \frac{1}{1-z^{-1}} = \frac{z}{z-1}$\\ \vspace{.07in}

Method 2:(1 step)\\

$E(s) = \frac{1}{s} \implies N(s) = 1, D(s) = s, p_1=0,D'(s) = 1$. Now, using Use \2'\\

$E(z) = \frac{1}{D'(0)}\frac{1}{1-z^{-1}e^{T0}} = \frac{1}{1-z^{-1}} = \frac{z}{z-1}$


\textcolor{red}{ Example 2:}\\

Given $E(s) = \frac{1}{s^2+3s+2} = \frac{1}{(s+1)(s+2)}$.  Find $E(z)$\\ \vspace{.05in}
Poles are:  $p_1 = -1$, and  $p_2 = -2$. Also, $N(s) = 1$ and
$D(s) = s^2+3s+2$. \\
Using \2'and $D'(s) = 2s+3$, we get, \\

$E(z) = \sum\limits_{n=1}^2 \frac{1}{D'(p_n)}\frac{1}{1-z^{-1}e^{Tp_n}} = \frac{1}{1-z^{-1}e^{-T}}-\frac{1}{1-z^{-1}e^{-2T}}$ \\




}



%----------- slide --------------------------------------------------%
\frame
{

\normalsize

\frametitle{Remarks}

\textcolor{red}{}\\

\begin{enumerate}
  \item If $G(s) = e^{-Ts}F(s) \implies G^*(s) = e^{-Ts}F^*(s)$\\
More generally,\\
$G(s)=E^*(s)F(s) \implies G^*(s) = E^*(s)F^*(s)$
	\item From \3' we saw that $E^*(\omega) = \frac{1}{T} \sum\limits_{n=-\infty}^{\infty} E(\omega+n\omega_s)$\\
where $\omega_s = \frac{2\pi}{T}$:  Sampling Frequency\\
That is, the spectrum of the sampled signal is scaled (by $\frac{1}{T}$) periodic extension of spectrum of $E(\omega)$. \\

\includegraphics[width=.85\linewidth]{./Figures/sample_graph2.png} \\%\vspace{-1in}\\


\end{enumerate}


}

%----------- slide --------------------------------------------------%
\frame
{

\normalsize

\frametitle{Sampling Theorem}

A finite energy signal $e(t)$ having a bandlimited spectrum $E(\omega)$ i.e. $E(\omega)=0 ~~\text{for} ~~|\omega|>|\omega_s|$ can be recovered (reconstructed) from its sampled values $e^*(t)$ (or $e(nT)$) by interpolation if  $\omega_s-\omega_m \ge \omega_m$ or $ \omega_s \ge 2\omega_m $(Nyquist frequency). In this case, no overlap occurs between periodic extensions and $E(\omega)$ or $e(t)$ can be uniquely recovered from $E^*(\omega)$ or $e^*(t)$ using a reconstruction low-pass filter (LPF) with frequency response:\\ \vspace{.1in}

$H(\omega) = \left \{
\begin{array}{lc}
	T& \omega \in R\\
	0& elsewhere
\end{array} \right.$


\begin{enumerate}
	\setcounter{enumi}{2}
\item For $\omega_s<2\omega_m,~ E^*(\omega)$ is no longer the replication of $E(\omega)$ and thus is no longer recoverable by LPF.  In this case, the upper frequencies in $E(\omega)$ get reflected into the lower frequencies in $E^*(\omega)$.  These frequencies are called ``fold over frequencies'' and the phenomenon is called``aliasing''.  Aliasing can be avoided by pre-filtering the signal first so that its bandlimit is less than one half of the sampling frequency.

\end{enumerate}


}
%Lecture 10 starts here

%----------- slide --------------------------------------------------%
\frame
{

\small

\frametitle{Some $s$-plane Properties of $E^*(s)$}
\vspace{-.8in}
\textcolor{red}{}\\
Recall from \3:\\
$E^*(s) =\frac{1}{T} \sum\limits_{n=-\infty}^\infty E(s+jn\omega_s)$ where $\omega_s = \frac{2\pi}{T}$ is the sampling frequency\\

Let $s=s+jm\omega_s, m\in I$\\
$E^*(s+jm\omega_s) = \frac{1}{T} \sum\limits_{m=-\infty}^{\infty} E(s+j(m+n)\omega_s)$\\
Change $m+n \to n'$\\
$E^*(s+jm\omega_s) = \frac{1}{T}\sum\limits_{n' = -\infty}^\infty E(s+jn'\omega_s) = E^*(s)$ \\ %\vspace{.2in}
Thus:\\
\begin{enumerate}
\item  $E^*(s)$ is periodic with period $\omega_s$\\ \vspace{.1in}

\item If $E(s)$ has a pole at $s=s_i$,\\
 then $E^*(s)$ has poles at $s_i \pm jm\omega_s$\\

\end{enumerate}

\textcolor{red}{Remark:}\\
  If Nyquist criterion is satisfied, i.e. poles of $E(s)$ are strictly within primary strip (i.e. $\omega_m < \omega_s/2$), sampling will result these poles to uniquely fold into complementary strips while in case of aliasing they fold back into the primary strip (not reconstructable).

\vspace{-2.2in}
\hspace{2.7in}
\includegraphics[width=.4\linewidth]{./Figures/poles_graph2.png}% \vspace{-1in}
}




\section{Reconstruction}
%----------- slide --------------------------------------------------%
\frame
{

\small

\frametitle{Reconstruction using Hold Devices}
The output of the digital controller must be converted back to a continuous-time signal before applying it to the plant. This
process is called signal reconstruction with hold devices. Note that ideal filters (infinite interpolation) cannot be used since
they are not realizable (not causal).
\begin{center}
\includegraphics[width=.4\linewidth]{./Figures/hold_diagram.png}
\end{center} \\
\textbf{Goal:}  Given $e(nT)~\forall n$, generate $\bar e(t)$ which is as close as possible to $e(t)$\\ \vspace{.1in}
Expand $e(t)$ at $t=nT$ using Taylor series:\\
$e(t) = e(nT) + e^{(1)}(nT)(t-nT) + e^{(2)}(nT)\frac{(t-nT)^2}{2!}+\hdots$\\
where $e^{(i)}(nT) = \frac{d^i e(t)}{dt^i}|_{t=nT}$. Using backward difference method \\ %\vspace{.1in}

$e^{(1)}(nT) = \frac{e(nT) -e((n-1)T)}{T}$. \\
$e^{(2)}(nT) = \frac{e^{(1)}(nT)-e^{(1)}((n-1)T)}{T} = \frac{e(nT) -2e((n-1)T) +e((n-2)T)}{T^2}$\\ \vspace{.1in}
That is, we can use all past and present $e(nT)$ samples to generate $\bar e(t)$ at $t=nT$ (interpolation)\\ \vspace{.1in}
Better approximation $\implies$ more terms $\implies$ more delays $\implies$ more sluggish and less stable



}
%----------- slide --------------------------------------------------%
\frame
{

\small

\frametitle{Zero-Order Hold (ZOH)}
\vspace{-.5in}
\begin{itemize}	\item ZOH implements only 1st term in expansion $\bar e(t) = e(nT), ~~~~~ \text{\small{when}}~~  nT \leq t< (n+1)T$ i.e holds the value $e(nT)$ until next sample arrives.
	\item No memory or delay is required for this hold device.

\end{itemize}
\includegraphics[width=.5\linewidth]{./Figures/ZOH_diagram.png}  \\
	 $g_{ho}(t) = u_s(t)-u_s(t-T)$\\
		$G_{ho}(s) = \mathscr{L}\{g_{ho}(t)\} = \frac{1}{s}-\frac{e^{-Ts}}{s} = \frac{1-e^{-Ts}}{s}$\\
Let $s=j\omega$\\
$G_{ho}(j\omega) = \frac{1-e^{-jT\omega}}{j\omega} $\\
$= \frac{T}{2}\frac{e^{jT\omega/2}-e^{-jT\omega/2}}{j\omega T/2}e^{-j\omega T/2}$ \\
$ = T \frac{sin \omega T/2}{\omega T/2} e^{-j\omega T/2} =Tsinc(\omega T/2)e^{-j\omega T/2}$\\
$ =\frac{2\pi}{\omega_s} sinc(\pi \omega /\omega_s)e^{-j \pi \omega/\omega_s}$\\

$|G_{ho}(j\omega)| = \frac{2\pi}{\omega_s}|sinc(\frac{\pi \omega}{\omega_s})|$\\
$\phi_{ho}(j\omega) = -\frac{\pi \omega}{\omega_s}+\theta,~~~~\theta = \left \{
\begin{array}{lr}
 	0&sinc(\frac{\pi \omega}{\omega_s})>0\\
	\pi&sinc(\frac{\pi \omega}{\omega_s})<0
\end{array} \right.$

\vspace{-2.5in}
\hspace{2.5in}
	\includegraphics[width=.4\linewidth]{./Figures/mag_plot.png}




}

%%----------- slide --------------------------------------------------%
\frame
{

\small


\frametitle{Zero-Order Hold}

Note: $G^*_{ho}(s)= (1-e^{-TS}) [\frac{1}{s}]^* = (1-e^{-TS}) \frac{1}{1-e^{-TS}}=1$ or $G_{ho}(z)=1$.  Why does this make sense? \\ \vspace{.06in}

\textcolor{red}{ Example:}\\

Consider $e(t) = 5sin(\omega t)$ sampled at $T= \frac{\pi}{6}$ (or $\omega_s=12~Rad/sec$)  and reconstructed with ZOH.  Find the magnitude and phase of $\bar e(t)$ for a) $\omega = 3~ Rad/sec$, b) $\omega=15~Rad/sec$\\ \vspace{.1in}
We have $\bar{E}(\omega)= E^*(\omega)G_{ho}(j\omega)$ where $E^*(\omega)$ is related to $E(\omega)$ as $E^*(\omega) = \frac{1}{T}\sum\limits_{n=-\infty}^{\infty} E(\omega +n \omega_s)$ and

a) $\omega = 3~ Rad/sec$ (below $\omega_s/2$ hence little attenuation) \\
$|G_{ho}(j\omega)|_{\omega = 3} = \frac{\pi}{6} sinc(\frac{\pi \omega}{\omega_s}) = \frac{\pi}{6}\frac{sin(\pi/4)}{\pi/4}$\\
$|\bar e(t)| = |e^*(t)G_{ho}(j3)|=5 \frac{sin(\pi/4)}{\pi/4} \approx 4.5 $ \\
$\phi_{ho}(j3) = -\pi \omega/\omega_s = -\pi/4$ \\ \vspace{.1in}

b) $\omega = 15~ Rad/sec$ (above $\omega_s$ hence more attenuation)\\
$|G_{ho}(j\omega)|_{\omega=15} = \frac{\pi}{6} sinc(\frac{\pi \omega}{\omega_s}) = \frac{\pi}{6} \frac{sin(5 \pi/4)}{5\pi/4}$\\
$|\bar e(t)|  \approx 0.9$

$\phi_{ho}(j15) = -\pi \omega/\omega_s +\pi = -5 \pi/4+\pi = -\pi/4$



}

%----------- slide --------------------------------------------------%
\frame
{

\renewcommand{\theenumi}{\alph{enumi}}
\frametitle{First-Order Hold}

First-order hold (FOH)implements the first two terms in the expansion. That is,
\\ \vspace{0.1in}

	$\bar e(t) = e(nT) + e^{(1)}(nT)(t-nT) ~~~~~ \text{when}~~ nT \leq t< (n+1)T$\\ \vspace{0.05in}
where	$e^{(1)}(nT) = \frac{[e(nT) -e((n-1)T)]}{T}$ \\ \vspace{0.05in}
Thus, the input-output equation of FOH is \\ \vspace{0.05in}
	$\bar e(t) = e(nT) + \frac{[e(nT) -e((n-1)T]}{T}(t-nT)$\\ \vspace{0.05in}

	For $ n=0$\\
$ \bar e(t) = e(0) + \frac{[e(0)-e(-T)]t}{T} = 1+\frac{[1-0]t}{T} = 1+t/T,~~0 \le t <T$\\ \vspace{0.05in}
	For $n=1 $\\
$\bar e(t) = e(1) + \frac{[e(1) - e(0)](t-T)}{T} =-\frac{1}{T}(t-T) = 1-t/T,~~T\le t<2T$\\ \vspace{0.05in}
	For $n>1$ $ \bar e(t) = 0,~~ t\ge 2T$\\ \vspace{.05in}
Thus,\\
$g_{h1}(t) = \left \{
\begin{array}{cc}
	1+t/T& 0 \le t<T\\	
	1-t/T &T \le t<2T\\
	0 & t \ge 2T
\end{array}  \right.$ \\



\vspace{-.7in}
\hspace{2.2in}
\includegraphics[width=.5\linewidth]{./Figures/foh_diagram.png} \\

}

%----------- slide --------------------------------------------------%
\frame
{

\frametitle{First-Order Hold}
\small
\vspace{-.2in}	
Alternatively, $g_{h1}(t)$ can be expressed as, \\
$g_{h1}(t) = (1+\frac{t}{T})u_s(t) - 2(1+\frac{t-T}{T}) u_s(t-T) +(1-\frac{t-T}{T})u_s(t-2T)$\\ \vspace{.07in}
Take Laplace transform and use  the shifting property:\\
$G_{h1}(s) = \frac{1+Ts}{T}\left(\frac{1-e^{-Ts}}{s}\right )^2$\\ \vspace{.07in}
\textbf{Note:}  Similar to ZOH $G^*_{h1}(s)=1$. \\
Let $s=j\omega$, then the magnitude and phase responses:\\ \vspace{.07in}
$G_{h1}(\omega) = \frac{1+j\omega T}{T}\left(\frac{1-e^{-j\omega T}}{j\omega}\right )^2$\\ \vspace{.09in}
$|G_{h1}(\omega)| =\frac{2\pi}{\omega_s} \sqrt{1+\frac{4\pi^2\omega^2}{\omega_{s}^2}}sinc^2(\pi \omega/\omega_s)$\\ \vspace{.09in}
$\phi_{h1}(\omega) = \tan^{-1} \frac{2\pi \omega}{\omega_s}-\frac{2\pi \omega}{\omega_s}$\\ \vspace{.09in}
\textcolor{red}{Remarks:}\\
\begin{itemize}
	\item FOH better than ZOH for $\omega < \omega_s/2$
	\item ZOH better for $\omega \ge \omega_s/2$
	\item ZOH has linear phase (better)
	\item FOH uses one delay $\implies$ less stable
\end{itemize}
\vspace{-2in}
\hspace{3in}
\includegraphics[width=.3\linewidth]{./Figures/foh_plot2.png}



}




%\includegraphics[width=.16\linewidth]{./Figures/sampling_graphs.png}








\end{document}

