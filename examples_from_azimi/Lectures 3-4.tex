%----------- Necessary Style Preamble -----------%

\documentclass[mathserif, 10pt]{beamer} %

\usepackage[framesassubsections]{beamerprosper}
\usepackage{beamerthemesplit} % Activate for custom appearance
\usepackage[english]{babel}
\usepackage[latin1]{inputenc}
\usepackage{amsmath, hyperref, subfigure, multirow, rotating}
\usepackage{epstopdf}
\usepackage{verbatim}
\usepackage{listings}
\usepackage{color}
\usepackage{esint}

\usetheme{Frankfurt}
\usecolortheme{default}
\usecolortheme[rgb={0.1,0.4,0.0}]{structure} % CSU color style
%\usecolortheme[rgb={0.541,0.149,0.196}]{structure} % BME color style

\setbeamersize{text margin left=0.5cm}
\setbeamersize{text margin right=0.5cm}

\setbeamercovered{transparent}

\usepackage{remreset}
\makeatletter
\@removefromreset{subsection}{section}
\makeatother
\setcounter{subsection}{1}

\makeatletter
\newcommand\xleftrightarrow[2][]{\ext@arrow 0099{\longleftrightarrowfill@}{#1}{#2}}
\def\longleftrightarrowfill@{\arrowfill@\leftarrow\relbar\rightarrow}
\makeatother


\newcommand\Wider[2][3em]{%
\makebox[\linewidth][c]{%
  \begin{minipage}{\dimexpr\textwidth+#1\relax}
  \raggedright#2
  \end{minipage}%
  }%
}

%----------- Math Definitions -----------%

\definecolor{Blue}{rgb}{0,0,1}
\def\ci{\perp\!\!\!\perp}
\def\a{\mathbf{a}}
\def\d{\mathbf{d}}
\def\e{\mathbf{e}}
\def\b{\mathbf{b}}
\def\q{\mathbf{q}}
\def\v{\mathbf{v}}
\def\x{\mathbf{x}}
\def\y{\mathbf{y}}
\def\z{\mathbf{z}}
\def\D{\mathbf{D}}
\def\S{\mathbf{S}}
\def\X{\mathbf{X}}
\def\Z{\mathbf{Z}}
\def\balph{\boldsymbol\alpha}
\def\bmu{\boldsymbol\mu}
\def\btheta{\boldsymbol\theta}
\def\bomega{\boldsymbol\omega}
\def\bDelta{\boldsymbol\Delta}

%----------- Title Page Parameters -----------%
\setbeamercovered{transparent}

\title[Digital Control \& Digital Filters]{Digital Controls \& Digital Filters \\ Lectures 3 \& 4}
\author[M.R. Azimi]{M.R. Azimi, Professor}
\institute[CSU-ECE]{Department of Electrical and Computer Engineering \\ Colorado State University}
\date{Spring 2017}

\logo{\includegraphics[height=0.5cm]{csu-logo.png}}

\newcommand{\unt}[1]{ \mathrm{\ #1}}


\begin{document}

\frame{\titlepage}



\section{Inverse $z$-Transform}

%----------- slide --------------------------------------------------%
\frame
{

\normalsize

\frametitle{Inverse $z$-Transform Methods}


Two methods are covered here.

\vspace{.1in}
\textcolor{blue}{1-Partial Fraction Expansion (PFE)}

This method is parallel to partial fraction expansion used in the Laplace transform, with one minor modification.  In this case, we expand $\frac{X(z)}{z}$ instead of $X(z)$, (since unlike $\mathcal{L}^{-1}\{\frac{1}{s+a}\}=e^{-at}$  for $z$-transform $\mathcal{Z}^{-1}\{\frac{1}{z+a} \}$ does not exist in the tables)\\ \vspace{.2in}

\vspace{.05in}
\textcolor{red}{Example:}  \\

Given $X(z) = \frac{z^2+z}{(z-1)^2}$ determine $x(n)$ using PFE.\\
$\frac{X(z)}{z}=\frac{z+1}{(z-1)^2}=\frac{A_1}{(z-1)}+\frac{A_2}{(z-1)^2}$\\
$A_1 = \frac{d}{dz} \{(z-1)^2 \frac{X(z)}{z} \}|_{z=1}=1$\\
$A_2 = (z-1)^2 \frac{X(z)}{z}|_{z=1}=2$\\ %\vspace{.2in}
Hence:\\ %\vspace{.2in}
$X(z) = \frac{z}{z-1}+\frac{2z}{(z-1)^2}$\\ \vspace{.1in}
$x(n) = \mathcal{Z}^{-1} \{ \frac{z}{z-1}\} +\mathcal{Z}^{-1} \{ \frac{2z}{(z-1)^2} \} = (1+2n)u_s(n)$\\


}


%----------- slide --------------------------------------------------%
\frame
{

\normalsize

\frametitle{Inverse $z$-Transform Methods-Cont.}

\textcolor{blue}{2-Inversion Formula Method}

\vspace{.05in}
If $\{ x(n) \}$ has $z$-transform $X(z) = \sum\limits_{n=0}^{\infty} x(n) z^{-n}$ \\ \vspace{.2in}
then $x(n)$ can be recovered from $X(z)$ by the inverse integral formula:\\ \vspace{.2in}
$x(n) = \frac{1}{2 \pi j}  \ointctrclockwise_C X(z) z^{n-1} dz$\\ \vspace{.2in}
$C$: Any closed contour in ROC of $X(z)$.\\ \vspace{.2in}
$\ointctrclockwise_C$: Line or contour integral along $C$ in counterclockwise direction. \\ \vspace{.2in}
The inversion integral may easily be evaluated using Cauchy's residue theorem.\\ %\vspace{.2in}


}

%----------- slide --------------------------------------------------%
\frame
{

\normalsize

\frametitle{Inversion Formula Method}


\textcolor{blue}{Cauchy's Residue Method} \\

\vspace{.05in}
Let $F(z)$ be an analytic function in shaded region $R$ (excludes singularities of $F(z)$) enclosed by contour $C$.  According to Cauchy theorem the integral of $F(z)$ over $R$ is zero or, we have\\
\vspace{.05in}
\[\ointctrclockwise_c F(z) dz = 2 \pi j \sum\limits_{i=1}^n \ointctrclockwise_{c_i} F(z) dz= 2 \pi j \sum\limits_{i=1}^n r_i\] \\

\vspace{.05in}
where $r_i = \ointctrclockwise_{c_i} F(z) dz$ is the $i^{th}$ Residue \\
associated with $z=z_i$ pole of $F(z)$.




\vspace{-.47in}
\begin{figure}[h!]
\centering
\hspace{2in}
\includegraphics[width=.4\textwidth]{./figures/residue.png}
%\caption{\label{fig:residue} }
\end{figure}




}
%----------- slide --------------------------------------------------%
\frame
{

\normalsize

\frametitle{Inversion Formula Method-Cont.}

\textcolor{blue}{Residue Theorem for Inverse $z$-Transform (IZT)}\\

\vspace{.05in}
Given $X(z),|z|>R$, the corresponding IZT can be found by evaluating the integral:\\ \vspace{.2in}
$x(n) = \frac{1}{2 \pi j} \ointctrclockwise_c X(z) z^{n-1}dz = \sum Res$ of $X(z)z^{n-1}$ corresponding to the poles of $X(z)z^{n-1}$ that lie inside $C$ that encloses $|z|=R$.\\ \vspace{.2in}

The residue of $X(z)z^{n-1}$ at a given pole, $z=z_i$ with multiplicity $m$, can be calculated using:\\
$Res_{z=z_i}=\frac{d^{m-1}}{dz^{m-1}} \left [\frac{(z-z_i)^m}{(m-1)!} X(z)z^{n-1} \right ] \bigg |_{z=z_i}$\\

For example:\\
m=1:~~~~$Res_{z=z_i} = (z-z_i)X(z)z^{n-1} |_{z=z_i}$\\
m=2:~~~~$Res_{z=z_i} = \frac{d}{dz} [(z-z_i)^2X(z)z^{n-1}] |_{z=z_i}$\\


}




%----------- slide --------------------------------------------------%
\frame
{
\frametitle {Inversion Formula Method-Cont.}

\textcolor{red}{Example: } \\ %\vspace{.2in}

Find IZT of $X(z) = \frac{1}{(z-1)(z-0.5)}$ using the residue method. \\ %\vspace{.2in}


\vspace{.05in}

Since $X(z) z^{n-1} = \frac{z^{n-1}}{(z-1)(z-0.5)}$ has a pole at $z=0$ when $n=0$ but not for $n> 0$, the two cases should be considered separately.\\
\textcolor{red}{Case 1:}  n=0, ~~$X(z)z^{n-1} = \frac{1}{z(z-1)(z-0.5)}$\\
Residue theorem gives: $x(0)=Res_{z=0}+Res_{z=1}+Res_{z=0.5}$\\
$Res_{z=0} = zX(z)z^{n-1}|_{z=0}=2$\\
$Res_{z=1} = (z-1)X(z)z^{n-1}|_{z=1}=2$\\
$Res_{z=0.5} = (z-0.5)X(z)z^{n-1}|_{z=0.5}=-4$\\
Thus, $x(0)=0$

\textcolor{red}{Case 2:}  $n\ge 1$,~~$x(n) = Res_{z=1}+Res_{z=0.5}$\\
$Res_{z=1} = (z-1)X(z)z^{n-1}|_{z=1}=2$\\
$Res_{z=0.5} = (z-0.5)X(z)z^{n-1}|_{z=0.5}=-2(0.5)^{n-1}$\\
Thus, $x(n) = 2-2(0.5)^{n-1}, n \ge 1$\\ \vspace{.1in}

Combine both results:

\vspace{-.5in}

\begin{displaymath}
\hspace{2in}
   x(n) = \left\{
    \begin{array}{lr}
       0 &  n=0\\
       2(1-(0.5)^{n-1}) &  n \ge 1
    \end{array}
   \right.
\end{displaymath}

}

\frame
{

\frametitle{Inversion Formula Method: Special Case}
\normalsize

Assume $C$ is confined to the unit circle i.e. $z=e^{j\Omega}$, then $dz=j d \Omega e^{j\Omega}$ and the
inversion formula becomes

\[x(n) = \frac{1}{2 \pi j}  \ointctrclockwise_C X(z) z^{n-1} dz = \frac{1}{2 \pi} \int^\pi_{-\pi} X(e^{j\Omega}) e^{j\Omega n} d\Omega \]

which is nothing but the inverse discrete time Fourier transform (DTFT).  This shows the relation between $z$-transform
and DTFT.  The forward transform is then

\[X(z)_{z=e^{j\Omega}} = X(e^{j\Omega})= \sum\limits_{n=0}^{\infty} x(n) e^{-j\Omega n} \]


}
\section{Representation of LTI systems}

%----------- slide --------------------------------------------------%
\frame
{

\normalsize

\frametitle{Representation of Discrete LTI Systems}
A discrete-time LTI system can be described by one of the following formulations:\\ \vspace{.2in}
\vspace{-.15in}
\begin{enumerate}
	\item Difference equation-General: Complete solution
	\item Convolution summation-Only particular solution (or forced response).
	\item Transfer function-Only particular solution (or forced response).
	\item State-Space equation-General: Complete solution
\end{enumerate}

\textcolor{blue}{1-Difference Equation and Transfer Function Representations}

A discrete-time LTI system can be represented by the following $N^{th}$ order constant coefficient difference equation:\\
\[\sum\limits_{k=0}^N a_k y(n-k) = \sum\limits_{\ell=0}^M b_{\ell} u(n-\ell), ~~~~a_0 = 1\]

where $y(n)$: output, $u(n)$: input,  $a_k$s and $b_l$s: coefficients. Alternatively,
\[\underbrace{y(n)}_{present~output} = \underbrace{\sum\limits_{\ell = 0}^M b_{\ell} u(n-\ell)}_{past~ \& present~inputs}-\underbrace{\sum\limits_{k=1}^N a_k y(n-k)}_{past~outputs}\]

}


%----------- slide --------------------------------------------------%
\frame
{

\normalsize

\frametitle{Difference Equation-Special Cases}

\textcolor{red}{Case a:} Consider the case where $N=0$ and $M \ne 0$ then we get\\
$y(n) = \sum\limits_{\ell=0}^M b_{\ell} u(n-\ell)$ \\
i.e. no recursion and hence the system is called \textit{Nonrecursive} system. \\

Taking the $z$-transform of both sides gives,\\
$Y(z) = (\sum\limits_{\ell=0}^M b_{\ell} z^{-\ell}) U(z)$, Assuming zero IC's\\
which gives transfer function:
$H(z) = \frac{Y(z)}{U(z)}=\sum\limits_{\ell = 0}^M b_{\ell} z^{-\ell}$\\

i.e. the transfer function contains all zeros, hence also called \textit{All-Zero }systems  \\

Comparing $H(z) =\sum\limits_{\ell = 0}^M b_{\ell} z^{-\ell}$ with  $H(z) = \sum\limits_{\ell =0}^{\infty} h(\ell) z^{-\ell}$ we note that \\

 $h(\ell) = b_{\ell}$ for $\ell \in[0, M]$ and $h(\ell) = 0~~~\forall \ell >M$\\

i.e. the impulse response is finite duration hence also called \textit{Finite Impulse Response (FIR)}. Alternatively we have:\\
$y(n) = \sum\limits_{\ell = 0}^M h(\ell) u(n-\ell)$\\

This is called a \textit{Moving Average (MA)} process.
}

%----------- slide --------------------------------------------------%

\frame
{

\normalsize
\frametitle{Difference Equation-Special Cases}

\textcolor{red}{Case b:} Consider the case where $N \ne 0$ and $M = 0$, then \\
$y(n) = b_0 u(n) - \sum\limits_{k=1}^N b_k y(n-k)$\\
i.e. the recursive terms are still there.  Taking the $z$-transform of both sides gives:\\
$Y(z) = b_0U(z)-\sum\limits_{k=1}^N a_{k} z^{-k} Y(z) \Longrightarrow H(z) = \frac{Y(z)}{U(z)}=\frac{b_0}{\sum\limits_{k=0}^N a_{k} z^{-k}}$\\
i.e. an \textit{All-Pole system}.  Compare to $H(z) = \sum\limits_{n=0}^\infty  h(n) z^{-n}$, it is clear that $h(n)\neq 0 ~\forall n\geq0$
i.e. \textit{infinite impulse response (IIR)}.

\vspace{.05in}
\textcolor{red}{Case c (general)}: Here $N \ne 0$ and $M \ne 0$.  Thus, the transfer
function becomes, \\
$H(z) = \frac{Y(z)}{U(z)} = \frac{\sum\limits_{\ell = 0}^{M} b_{\ell} z^{-\ell}}{\sum\limits_{k=0}^N a_k z^{-k}} = \frac{B(z)}{A(z)}$\\
which has both poles and zeros and is the general case of \textit{Recursive or IIR} systems.

}

%----------- slide --------------------------------------------------%
\frame
{

\normalsize

\textcolor{red}{Remark:}
\begin{enumerate}
	\item The rational transfer function, $H(z)$, is \textit{proper} if the order of the numerator polynomial (in $z$) is less than or equal to that of the denominator polynomial (i.e. $M \ge N$) and
\textit{strictly proper} if it is less than the order of the denominator polynomial (i.e. $M > N$).
 	\item Not proper $\implies$ Noncausal $\implies$ Nonrealizable. \\
\end{enumerate}
\vspace{.1in}

e.g., $H(z) = \frac{z^2+z}{z-1}=\frac{Y(z)}{U(z)} \implies (z-1)Y(z) = (z^2+z)U(z)$\\
$\implies y(n+1) = y(n) + u(n+2) +u(n+1) \implies$ noncausal due to $y(n+1)$ depending on $u(n+2)$. \\

Alternatively, using long division,\\

$H(z) = \frac{z^2+z}{z-1}= z+2+2z^{-1}+\cdots$\\
Comparing this with $H(z) = \sum\limits_{n=-\infty}^\infty h(n) z^{-n}$

it is clear that $h(-1)=1 \implies h(n) \ne 0~for~n<0$ i.e. noncausal.\\

}


\frame
{

\normalsize

\frametitle{Realization of Discrete-Time LTI Systems and SFG}

A discrete-time LTI system described by difference equation,\\
$y(n) = \sum\limits_{\ell=0}^M b_{\ell} u(n-\ell) - \sum\limits_{k=1}^N a_k y(n-k)$\\
can be realized on a hardware structure in a variety of forms using three basic elements.
\begin{enumerate}
\item Multipliers \\ \vspace{.07in}
	\item Delays     \\ \vspace{.07in}
	\item Adders	\\
\vspace{-.75in}
\begin{figure}[h!]

\hspace{1in}
\includegraphics[width=.4\linewidth]{./Figures/sfg_dts.png}
\end{figure}
	
A direct form realization of difference equation is shown below.\\

\begin{figure}[h!]
\includegraphics[width=.7\linewidth]{./Figures/sim_diag2.png}
\end{figure}


\end{enumerate}


}

%----------- slide --------------------------------------------------%
\frame
{

\normalsize

\frametitle{Realization of Discrete-Time LTI Systems and SFG}
\vspace{-.1in}
The signal flow graph (SFG) version is also shown.

\begin{figure}[h!]
\includegraphics[width=.6\linewidth]{./Figures/sfg_sim_diag.png}
\end{figure}

\textcolor{red}{Example 1:} Derive the difference equation for a discrete differentiator. \\

\textcolor{blue}{Solution:}  Analog differentiator is $y(t) = \frac{du(t)}{dt}$. To discretize use backward difference method
$y(t) \approx \frac{u(t)-u(t-\Delta)}{\Delta}$ and let $t=nT$ and $\Delta=T$  \\

$y(nT) = \frac{u(nT)-u((n-1)T)}{T}$\\
or simply $y(n) = \frac{u(n)-u(n-1)}{T}$ which is difference equation of discrete differentiator. Transfer function is \\
\vspace{.1in}
$H(z)=\frac{Y(z)}{U(z)} = \frac{1-z^{-1}}{T} = \frac{z-1}{zT}$
\vspace{-.43in}
\begin{figure}[h!]
\hspace{2in}
\includegraphics[width=.45\linewidth]{./Figures/backward_difference2.png}
\end{figure}

}
%----------- slide --------------------------------------------------%
\frame
{

\normalsize

\frametitle{Solution of Difference Equations using $z$-transform}
To find the complete solution (i.e. homogenous+particular) of a discrete LTI system: (a) take $z$-transform and impose ICs (b) apply PFE (or residue method) to find $y(n)$ for a given input $u(n)$.

\vspace{0.05in}
\textcolor{red}{Example 2:}\\
Given $y(n+2) +3y(n+1) +2y(n) = u(n), ~~ u(n) = \delta(n),~~ y(0)=1,y(1)=-1$\\
Find the \textit{complete response} of the system.\\
\textcolor{blue}{Solution:}\\ Recall form properties:  $ \mathcal{Z} \{x(n+n_0) \} = z^{n_0}[X(z)- \sum\limits_{k=0}^{n_0-1} x(k) z^{-k}]$. \\
Then, \\
$z^2[Y(z)-y(0)-y(1)z^{-1}]+3z[Y(z)-y(0)] +2Y(z)  = U(z)$\\ \vspace{.1in}
$\implies (z^2+3z+2)Y(z) = U(z)+y(0)z^2 +y(1)z + 3zy(0)$ \\
But $U(z)=1$, then \\ \vspace{.1in}
$Y(z) = \frac{1}{z^2+3z+2}+\frac{z^2+2z}{z^2+3z+2} = Y_{p}(z)+ Y_{h}(z)= \frac{z^2+2z+1}{z^2+3z+2}=\frac{z+1}{z+2}$\\ \vspace{.1in}
}


%----------- slide --------------------------------------------------%
\frame
{

\normalsize

\frametitle{Solution of Difference Equations using $z$-transform-Cont.}

Using PFE, expand $\frac{Y(z)}{z}$\\ \vspace{.1in}
$\frac{Y(z)}{z} = \frac{z+1}{z(z+2)} = \frac{1/2}{z}+\frac{1/2}{z+2}$\\ \vspace{.1in}
$y(n) = \frac{1}{2} \delta(n) +\frac{1}{2} (-2)^nu_s(n)$\\

Note that the particular solution is $Y_p(z)= H(z)U(z)$ (here $H(z)=\frac{1}{z^2+3z+2}$) while homogeneous solution $Y_h(z)$ only depends on IC's. \\ \vspace{.1in}

If needed, we can find these responses separately, as follows

$\frac{Y_{h}(z)}{z}= \frac{z+2}{z^2+3z+2}=\frac{1}{(z+1)} \longrightarrow y_h(n)= (-1)^n u_s(n)$ \\ \vspace{.1in}

While 

$\frac{Y_{p}(z)}{z}= \frac{1}{z(z^2+3z+2)}=\frac{1/2}{z}-\frac{1}{(z+1)}+\frac{1/2}{(z+2)}$ \\ \vspace{.1in}

Thus, \\
 $y_p(n)= \frac{1}{2} \delta(n) -(-1)^n u_s(n)+\frac{1}{2} (-2)^nu_s(n)$ \\ \vspace{.1in}

}


\end{document}

