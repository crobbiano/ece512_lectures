%----------- Necessary Style Preamble -----------%

\documentclass[mathserif, 10pt]{beamer} %

\usepackage[framesassubsections]{beamerprosper}
\usepackage{beamerthemesplit} % Activate for custom appearance
\usepackage[english]{babel}
\usepackage[latin1]{inputenc}
\usepackage{amsmath, hyperref, subfigure, multirow, rotating}
\usepackage{epstopdf}
\usepackage{verbatim}
\usepackage{listings}
\usepackage{color}
\usepackage{esint}
%\usepackage{enumitem}

\usetheme{Frankfurt}
\usecolortheme{default}
\usecolortheme[rgb={0.1,0.4,0.0}]{structure} % CSU color style
%\usecolortheme[rgb={0.541,0.149,0.196}]{structure} % BME color style

\setbeamersize{text margin left=0.5cm}
\setbeamersize{text margin right=0.5cm}

\setbeamercovered{transparent}

\usepackage{remreset}
\makeatletter
\@removefromreset{subsection}{section}
\makeatother
\setcounter{subsection}{1}

\makeatletter
\newcommand\xleftrightarrow[2][]{\ext@arrow 0099{\longleftrightarrowfill@}{#1}{#2}}
\def\longleftrightarrowfill@{\arrowfill@\leftarrow\relbar\rightarrow}
\makeatother


\newcommand\Wider[2][3em]{%
\makebox[\linewidth][c]{%
  \begin{minipage}{\dimexpr\textwidth+#1\relax}
  \raggedright#2
  \end{minipage}%
  }%
}

%----------- Math Definitions -----------%

\definecolor{Blue}{rgb}{0,0,1}
\def\ci{\perp\!\!\!\perp}
\def\a{\mathbf{a}}
\def\d{\mathbf{d}}
\def\e{\mathbf{e}}
\def\f{\mathbf{f}}
\def\g{\mathbf{g}}
\def\b{\mathbf{b}}
\def\q{\mathbf{q}}
\def\v{\mathbf{v}}
\def\x{\underline{x}}
\def\y{\mathbf{y}}
\def\u{\underline{u}}
\def\z{\underline{z}}
\def\D{\mathbf{D}}
\def\S{\mathbf{S}}
\def\X{\underline{X}}
\def\Z{\mathbf{Z}}
\def\balph{\boldsymbol\alpha}
\def\bmu{\boldsymbol\mu}
\def\btheta{\boldsymbol\theta}
\def\bomega{\boldsymbol\omega}
\def\bDelta{\boldsymbol\Delta}

%----------- Title Page Parameters -----------%
\title[Digital Control \& Digital Filters]{Digital Controls \& Digital Filters \\ Lectures 7 \& 8}
\author[M.R. Azimi]{M.R. Azimi, Professor}
\institute[CSU-ECE]{Department of Electrical and Computer Engineering \\ Colorado State University}
\date{Spring 2017}

\logo{\includegraphics[height=0.5cm]{csu-logo.png}}

\newcommand{\unt}[1]{ \mathrm{\ #1}}

\begin{document}

%----------- slide --------------------------------------------------%

\frame{\titlepage}



\section{Solutions of LTI Systems: State-Space Models}

%----------- slide --------------------------------------------------%
\frame
{

\normalsize

\frametitle{1. Time Domain Solution}


Consider an LTI system described by the following state-space equations, \\ $ \left\{
\begin{array}{l}
	\x(n+1) = A\x(n)+Bu(n)\\
	y(n) = C\x(n) +du(n)
\end{array} \right.$\\
%\vspace{0.1in}
Assume initial conditions,$\x(0)$, and input $u(n) ~\forall n \ge 0 $ are given.\\
Plugging for $n$ in the state-space equation gives\\
 $\x(1) = A \x(0)+Bu(0)$\\
$\x(2) = A \x(1)+Bu(1) = A^2\x(0) + AB u(0) + Bu(1)$\\
$ \x(3) = A \x(2) + Bu(2) = A^3 \x(0) + A^2Bu(0)+ABu(1)+Bu(2)$\\
.\\
$\x(n) = A^n  \x(0)+\sum\limits_{j=0}^{n-1} A^{n-1-j}Bu(j)$\\
The output will then be given by $y(n) =\underbrace{CA^n\x(0)}_{y_{zi}(k) = y_h(n)} + \underbrace{\sum\limits_{j=0}^{n-1}CA^{n-1-j}Bu(j) + du(n)}_{y_{zs}(n) = y_p(n)}$ \\ \vspace{.05in}
$y_{zi}(n):$ Zero-input response or homogeneous solution. \\
$y_{zs}(n):$ Zero-state response or particular solution.\\


}

%----------- slide --------------------------------------------------%
\frame
{

\normalsize

\frametitle{State Transition Matrix and Properties}

\vspace{-.05in}
Define $\phi(n) = A^n$ to be the \textit{state transition matrix}. This matrix has the following useful properties. \\
%\textcolor{red}{Properties of the state transition matrix} $\phi(k) = A^n$\\
\begin{enumerate}
\item  $\phi (0) = I$\\
\item $\phi(n+\ell) = A^{n+\ell}=A^nA^{\ell} = \phi(n)\phi(\ell)$\\
\item  $\phi(-n) = A^{-n} = (A^n)^{-1} = \phi^{-1}(n)$
\end{enumerate}
\textcolor{red}{Remarks}\\
\begin{enumerate}
	\item If $\x(0) = \underline{0}$ and $u(n) = \delta(n)$ then\\
$h(n) = CA^{n-1}B+d\delta(n)$

\item Using the expression for $h(n)$ the zero-state response can be rewritten as\\
$y_{zs}(n)= \sum\limits_{j=0}^{n-1}CA^{n-1-j}Bu(j) + du(n)= \sum\limits_{j=0}^{n} h(n-j)u(j)$\\
i.e. the convolution sum.
	\item State transition matrix for the diagonal case is:\\
$\phi(n) = \left[
\begin{array}{ccccc}
	\gamma_1^n  &   0                  &      \hdots   &        0\\
	0                     &\gamma_2^n  &\hdots           &0\\
	\vdots             &\vdots             &\ddots           &\vdots\\
	0                     &\hdots             &0&\gamma_N^n
\end{array} \right]$\\ \vspace{.1in}
\vspace{-.1in}
where $\gamma_i$'s are the poles of the system.

\newcounter{enumTemp}
    \setcounter{enumTemp}{\theenumi}
\end{enumerate}


}


%----------- slide --------------------------------------------------%
\frame
{

\small
\renewcommand{\theenumi}{\alph{enumi}}
\frametitle{Time Domain Solution}

%\begin{enumerate}
%\setcounter{enumi}{\theenumTemp}
%
%\item  test
%
%\setcounter{enumTemp}{\theenumi}
%\end{enumerate}

\textcolor{red}{Example:} A digital controller is described by $y(n+2)-0.7y(n+1) +0.1y(n) = 3u(n+1)$. \\

(a) Find $h(n)$ using the transfer function\\
(b) Convert to PVCF and another state-space representation\\
(c) Find $h(n)$ using state-space approach \\ %\vspace{.1in}
\begin{enumerate}
\item Take the $z$-transform and assume all IC's=0.
$H(z) = \frac{Y(z)}{U(z)} = \frac{3z}{z^2-0.7z+0.1}$  \\
Expand $\frac{H(z)}{z} = \frac{3}{(z-0.5)(z-0.2)} = \frac{10}{z-0.5}-\frac{10}{z-0.2}$
$h(n) = [10(0.5)^n-10(0.2)^n]u_s(n)$\\

\item PVCF matrices are:\\
$A = \left[
\begin{array}{cc}
	0&1\\
	-0.1&0.7
\end{array} \right] B = \left[
\begin{array}{c}
	0\\
	1
\end{array} \right] C = \left[
\begin{array}{cc}
	0&3
\end{array} \right] $ and $d=0$\\

For parallel form, expand $H(z)$ \\
 $\frac{3z}{(z-0.5)(z-0.2)} = \frac{5}{z-0.5}-\frac{2}{z-0.2}$ i.e. $\gamma_1 = 0.5, \gamma_2 = 0.2$\\
$x(n+1) = \left[
\begin{array}{cc}
	0.5&0\\
	0&0.2
\end{array} \right] x(n) + \left[
\begin{array}{c}
	1\\
	1
\end{array} \right] u(n)$\\
$y(n) = [5~-2] x(n)$
	\item $A^n = \left[
\begin{array}{cc}
	0.5^n&0\\
	0& 0.2^n
\end{array} \right],~h(n) = CA^{n-1}B = 5(0.5)^{n-1}-2(0.2)^{n-1}~\forall n \ge 0 $\\
$= 10(0.5)^n-10(0.2)^n~ \forall n \ge 0$
\end{enumerate}


}


%----------- slide --------------------------------------------------%
\frame
{

\normalsize

\frametitle{2. $z$-Transform Domain Solution}

Consider an LTI system given by the following state equations. \\
  $\left\{
\begin{array}{l}
	\x(n+1) = A\x(n)+Bu(n)\\
	y(n) = C\x(n)+du(n)
\end{array} \right. $ %\vspace{.2in}

We want to find expression for $Y(z)$.  Take the $z$-transform and impose IC's.\\ \vspace{.08in}
$z\left[\X(z)-\x(0)\right] = A\X(z) +BU(z)$\\
$(zI-A)\X(z) = z\x(0)+BU(z)$\\
$\X(z) = (zI-A)^{-1}[z\x(0)+BU(z)]$\\ \vspace{.1in}
Plugging the above equation in the output equation in the $z$ domain, \\ \vspace{.05in}
$Y(z) = C\X(z)+dU(z) = C(zI-A)^{-1}[z\x(0)+BU(z)]+dU(z)$\\
$=\underbrace{C(zI-A)^{-1}z\x(0)}_{Y_{zi}
(z)}+\underbrace{[C(zI-A)^{-1}B+d]U(z)}_{Y_{zs}(z)}$\\ \vspace{.1in}

Comparing to the time domain it is easily seen that:\\
\begin{enumerate}
	\item
$\mathcal{Z}\{A^n\} = (zI-A)^{-1}z$ which can be used to find $\phi(n) = A^n$. \\
\item Also from the zero-state response $Y_{zs}(z) = H(z)U(z)$ we arrive at:\\
$H(z) = C(zI-A)^{-1}B+d$


\end{enumerate}






}

%----------- slide --------------------------------------------------%
\frame
{

\renewcommand{\theenumi}{\alph{enumi}}
\frametitle{$z$-Transform Solution}

\textcolor{red}{Example 1: } Consider the system given by:\\
$\x(n+1) = \left[
\begin{array}{cc}
	0&1\\
	-2&-3
\end{array} \right] \x(n) + \left[
\begin{array}{c}
	1\\
	1
\end{array} \right] u(n)$ ,  ~~
$y(n) = \left[
\begin{array}{cc}
	-2 & 1
\end{array} \right] \x(n)$\\
(a) Find $H(z)$\\
(b) Find the state transition matrix $A^n$ and the zero-input solution for $\x(0)=\left[
\begin{array}{cc}
	0 & 1
\end{array} \right]^T$  \\
(c) Zero-state response to a unit step input. 
\begin{enumerate}
\item $H(z) =\frac{Y(z)}{X(z)} = C(zI-A)^{-1}B+d$\\
$(zI-A) = \left[
\scriptsize{\begin{array}{cc}
	z&-1\\
	2&z+3
\end{array}} \right],~(zI-A)^{-1} = \frac{\left[
\scriptsize{\begin{array}{cc}
	 z+3&1\\
	-2&z
\end{array}} \right]}{z^2+3z+2}$

$H(z) = \frac{Y(z)}{X(z)} = \left[
\begin{array}{cc}
	-2& 1
\end{array} \right] \frac{\left[
\scriptsize{\begin{array}{cc}
	 z+3&1\\
	-2&z
\end{array}} \right]}{z^2+3z+2} \left[
\begin{array}{c}
	1\\
	1
\end{array} \right] = \frac{-z-10}{z^2+3z+2}$\\

\item $A^n = \mathcal{Z}^{-1}\{(zI-A)^{-1}z\} = \mathcal{Z}^{-1}\left\{ \frac{\left[
\scriptsize{\begin{array}{cc}
	z(z+3)&z\\
	-2z&z^2
\end{array}} \right]}{z^2+3z+2} \right \}$\\

\end{enumerate}

}

\frame
{

\renewcommand{\theenumi}{\alph{enumi}}
\frametitle{$z$-Transform Solution}

\hspace{-.2in}
$A^n =\mathcal{Z}^{-1} \left\{ \left[
\begin{array}{cc}
\frac{2z}{z+1}-\frac{z}{z+2}&\frac{z}{z+1}-\frac{z}{z+2}\\
\frac{-2z}{z+1}+\frac{2z}{z+2}&\frac{-z}{z+1}+\frac{2z}{z+2}
\end{array} \right] \right\} = \left[
\scriptsize{\begin{array}{cc}
	2(-1)^n-(-2)^n&(-1)^n-(-2)^n\\
	-2(-1)^n+2(-2)^n& -1(-1)^n+2(-2)^n
\end{array}} \right], ~\forall n>0$ \\

\begin{enumerate}
	\setcounter{enumi}{2}

\item 

$y_{zi}(n)=CA^n\x(0) = \left[
\begin{array}{cc}
	-2& 1
\end{array} \right] \left[
\scriptsize{\begin{array}{cc}
	2(-1)^n-(-2)^n&(-1)^n-(-2)^n\\
	-2(-1)^n+2(-2)^n& -1(-1)^n+2(-2)^n
\end{array}} \right] \left[
\begin{array}{c}
	0\\
	1
\end{array} \right]=-3(-1)^n + 4(-2)^n, ~~\forall n\geq 0$ 

\item 

$Y_{zs}(z)=H(z) U(z)$. Here $U(z)=\frac{z}{z-1}$. Thus, \\ \vspace{.1in}

$Y_{zs}(z)=\frac{-z(z+10)}{(z-1)(z+1)(z+2)}$. \\ \vspace{.1in}
Applying PFE to $\frac{Y_{zs}(z)}{z}$ and taking inverse $z$-transform yields, \\ \vspace{.1in}

$y_{zs}(n)=(-\frac{11}{6}+\frac{9}{2}(-1)^n-\frac{8}{3}(-2)^n)u_s(n).$


\end{enumerate}
}

%----------- slide --------------------------------------------------%
\frame
{

\normalsize
\renewcommand{\theenumi}{\alph{enumi}}
\frametitle{Similarity Transformation}
For a given LTI system, state-space representation is not unique and all($\infty$) such representations are related via \textit{similarity transformations}. \\
Consider the following state-space representation where $\x(n)$ is the state vector, \\ \vspace{.05in}
 $\left\{
\begin{array}{rl}
	\x(n+1) &=A\x(n)+Bu(n)\\
	y(n)&=C\x(n)+du(n)
\end{array} \right.$
%\textcolor{red}{Example}\\
\\Let us define a new state vector $\z(n)$ where $\x(n) = T\z(n)$\\
$T$:  non-singular (inverse exists) transformation matrix\\
Plugging $\z(n)$ into these equations:\\ \vspace{.05in}

$\left\{
\begin{array}{rl}
	T\z(n+1) &= AT\z(n)+Bu(n)\\
	y(n)&=CT\z(n)+du(n)
\end{array} \right.$ \\
Now, multiplying by $T^{-1}$ gives a new state-space representation \\ \vspace{.05in}
$\left\{ \begin{array}{rl}
	\z(n+1)&=T^{-1}AT \z(n) + T^{-1}B u(n)\\
	y(n)& = CT \z(n)+du(n)
\end{array} \right.$ \\ \vspace{.06in}
with state vector $\z(n)$ and new matrices $\hat A = T^{-1}AT~~,~~\hat B = T^{-1}B~~,~~\hat C = CT~~,~~ \hat d=d$. \\



}
%----------- slide --------------------------------------------------%
\frame
{

\normalsize

\frametitle{Similarity Transformation}

\textbf{Note:} Since there are infinite possible $T$ matrices, there will be infinite number of state-space representations for a given system. \\ \vspace{.06in}


\textcolor{blue}{Invariance Theorem:} Transfer function $H(z)$ remains invariant under similarity transformation.\\ \vspace{.1in}
\textcolor{red}{Proof:}  Recall $\hat H(z) = \hat C(zI-\hat A)^{-1} \hat B +\hat d$\\ \vspace{.1in}
	We need to show that $\hat H(z) = H(z)$\\ \vspace{.1in}
Substituting for $\hat A = T^{-1}AT~~,~~\hat B = T^{-1}B~~,~~\hat C = CT~~,~~ \hat d=d$, we get\\ \vspace{.06in}
$\hat H(z) = CT(zI-T^{-1}AT)^{-1} T^{-1}B+d$\\
Now using $(P_1P_2)^{-1} = P_2^{-1}P_1^{-1}$ we can write, \\ \vspace{.1in}
$\hat H(z) = CT(zT-AT)^{-1}B+d=$\\
$C(zTT^{-1}-ATT^{-1})^{-1}B+d =$\\
 $C(zI-A)^{-1}B+d = H(z)$



}
%----------- slide --------------------------------------------------%
\section{Sampling and Reconstruction}

\frame
{

\small

\frametitle{Sampling and Reconstruction}
A typical digital control system is shown below.  To analyze such a system requires determining input-output relationships for sampler and hold/reconstruction devices.\\

\begin{center}
\includegraphics[width=.7\linewidth]{./Figures/Digital-block.png}
\end{center} \vspace{-.07in}

\textcolor{blue}{1. Ideal Sampler:}
\begin{itemize}
	\item Sampling time is much shorter than most
 significant time constant of plant (perfect switch).
 \item Between two consecutive sampling instants the sampler transmits no information i.e. two signals whose respective values at the sampling instants are equal will give the same sampled signals.
\end{itemize}
\includegraphics[width=.5\linewidth]{./Figures/switch.png} \\
Sampling function:\\
$\delta_T(t) = \sum\limits_{n=0}^{\infty} \delta(t-nT) = \delta(t) + \delta(t-T) + \delta(t-2T) + \hdots$ \vspace{.2in}



%\hspace{.3in}
%%



}
%----------- slide --------------------------------------------------%
\frame
{

\small

\frametitle{Sampling and Reconstruction}

\includegraphics[width=.55\linewidth]{./Figures/sampled-signal.png}\\ \vspace{.05in}
Output of sampler = $e(t)\delta_T(t) = e^{*}(t) = e(t) \sum\limits_{n=0}^{\infty} \delta(t-nT) = \sum\limits_{n=0}^\infty e(nT) \delta(t-nT)$ \\
Take Laplace transform to get:\\
  $E^*(s) = \sum\limits_{n=0}^\infty e(nT) e^{-nTS}$:  "Starred Transform" of $E(s)$ (or $e(t)$)\\ \vspace{.2in}
  Thus, starred transform is the Laplace transform of the sampled signal. \\ \vspace{.2in}
Replace $e^{Ts}=z$ or $TS = \ln z  \implies S = \frac{1}{T} \ln z$\\
$E^*(s)|_{s=\frac{1}{T}\ln z } = \sum\limits_{n=0}^\infty e(n) z^{-n} = E(z)$\\ \vspace{.2in}

Note that  $E(s)|_{s = \frac{1}{T}ln(z)} \ne E(z)$


}






\end{document}

